\documentclass{article}

\usepackage{graphicx}
\usepackage{mathtools}
\usepackage{minted}
\setminted[haskell]{escapeinside=@@}
\newcommand{\hs}{\mintinline{haskell}}
\usepackage[hidelinks]{hyperref}
\usepackage[utf8]{inputenc}
\usepackage[
backend=biber
]{biblatex}
\addbibresource{bibliography.bib}

\title{Algebraic Graphs with Class}
\author{
  Christoph Madlener\\
  \texttt{\href{mailto:madlener@in.tum.de}{madlener@in.tum.de}}
}

\begin{document}

\maketitle
\begin{abstract}
  \cite{mokhov2017algebraic} 
\end{abstract}

\section{Introduction}\label{sec:intro}
Graphs are a fundamental structure studied in depth by both mathematicians and
computer scientists alike. One very common definition states that a (directed)
graph is a tuple $G = (V,E)$, where $V$ is a set of vertices and $E \subseteq V
\times V$ is the set of edges. While this is a perfectly valid and natural
mathematical definition it is not necessarily suitable for implementation. This
can be illustrated by trying to directly translate this to Haskell:
\begin{minted}{haskell}
  data G a = G { vertices :: Set a, edges :: Set (a,a)}
\end{minted}
The value \hs{G {[1,2,3], [(1,2),(2,3)]}} then represents the graph $G =
(\{1,2,3\}, \{(1,2),(2,3)\})$, however \hs{G {[1,2], [(2,3)]}} does not
represent a consistent graph, as the edge refers to a non-existent node.
The state-of-the-art \texttt{containers} library implements graphs with
adjacency lists employing immutable arrays~\cite{king1995dfs}. The
consistency condition $E \subseteq V \times V$ is not checked statically though,
which can lead to runtime errors. \texttt{fgl}, another popular Haskell graph
library uses inductive graphs~\cite{erwig2001inductive}, also exhibiting partial
functions and potential for runtime errors due to the violation of consistency.

This lead Mokhov to conceive \textit{algebraic
  graphs}~\cite{mokhov2017algebraic}, a sound and complete representation for
graphs. They abstract away from graph representation details and characterize
graphs by a set of axioms. Algebraic graphs have a small safe core of graph
construction primitives and are suitable for implementing graph transformations.
These primitives are represented in the following datatype:
\begin{minted}{haskell}
  data Graph a = Empty
               | Vertex a
               | Overlay (Graph a) (Graph a)
               | Connect (Graph a) (Graph a)
\end{minted}
\hs{Empty} constructs the empty graph, \hs{Vertex} a graph with a single vertex
and no edges. \hs{Overlay} essentially is the union of two graphs and
\hs{Connect} additionally adds edges from all vertices from one graph to all
vertices of the other.

Alongside proper definitions for these primitives we will see
in~\autoref{sec:locale} that this is indeed a sound and complete graph
representation. We will also cover the algebraic structure (\ref{sec:algebra})
exhibited by these graphs and elegant graph transformations (\ref{sec:trafo})
based on a type class for the core. In~\autoref{sec:verification} we explore how
we can exploit the algebraic structure in verification using Isabelle/HOL.

In~\autoref{sec:deep} we will examine the deep embedding (i.e.\
the datatype as opposed to the type class of the preceding sections) regarding
applicability in practice and in verification. In~\autoref{sec:quotient} we
define a quotient type in Isabelle/HOL which in essence is a deep embedding
which satisfies the axioms.

\section{Type class/locale}\label{sec:locale}
\subsection{The Core}\label{sec:core}
\begin{itemize}
\item graph construction primitives
\item abstract from datatype \textrightarrow{} type class
\item fundamental functions (vertex, edge, etc.)
\end{itemize}
\subsection{Algebraic Structure}\label{sec:algebra}
\begin{itemize}
\item axioms (for digraphs)
\item subgraph - partial ordering
\item axioms for undirected graphs, other classes
\item instances in Haskell (directed, undirected, etc. via Eq instance)
\end{itemize}
\subsection{Graph Transformation Library}\label{sec:trafo}
\begin{itemize}
\item graph families (path, circuit, etc.)
\item transpose
\item functor
\item monad
\item removing edges
\end{itemize}
\subsection{Verification}\label{sec:verification}
\begin{itemize}
\item equational reasoning
\item reasoning in Isabelle/HOL \textrightarrow{} locale
  \begin{itemize}
  \item ``advanced'' graph concepts (walks, reachability, SCCs, etc.)
    \begin{itemize}
    \item more experimentation needed for actual feasibility (additional axioms required
      for e.g.\ $Vertex\ u \neq \epsilon$?)
    \item first impression: similar as with other representations; abstraction
      to more advanced concepts quickly hides representation
    \end{itemize}
  \item ``polymorphic'' lemmas
  \item compare to Kruskal AFP entry~\cite{Kruskal-AFP} (possibility to get algorithms for
    any graph representation which can instantiate locale)
  \item instantiation with pair\_digraph~\cite{GraphTheory-AFP} - soundness and
    completeness proof
  \end{itemize}
\end{itemize}

\section{Deep Embedding}\label{sec:deep}
\begin{itemize}
\item compact representation
\item compare to different concrete/executable representations wrt.\ efficiency
  in time and space 
  \begin{itemize}
  \item sparse graphs (edge list)
  \item adjacency map
  \end{itemize}
\item performance (\href{https://github.com/haskell-perf/graphs}{haskell-perf})
\item reasoning in Isabelle/HOL \textrightarrow{} induction is great
\item algorithms directly on deep embedding? \textrightarrow{} future work/open
  question (Haskell implementation converts to adjacency map for algorithms)
\end{itemize}
\subsection{Quotient Type}\label{sec:quotient}
\begin{itemize}
\item deep embedding which satisfies axioms
\item additional lemmas
\item minimization~\cite{mcconnell2005linear}
  \begin{itemize}
  \item interesting for quotient type, disconnect from Noschinski/any other
    representation for equality, same applies in Haskell for Eq instance)
    \item interesting for algorithms (assuming we can exploit compact
      representation)
    \end{itemize}
\end{itemize}

\section{Conclusion}\label{sec:conclusion}
\begin{itemize}
\item summarize findings
\item restate open questions, future work
\item labeled graphs (outlook)
\item related work: other algebraic approaches (semiring on
  matrices~\cite{dolan2013fun}, relational
  algebra~\cite{berghammer2020relational})
\item usage examples~\cite{mokhov2019language,beaumont2017high}
\end{itemize}
\printbibliography
\end{document}