\documentclass{article}

\usepackage{graphicx}
\usepackage{mathtools}
\usepackage[hidelinks]{hyperref}
\usepackage[utf8]{inputenc}
\usepackage[
backend=biber
]{biblatex}
\addbibresource{bibliography.bib}

\title{Algebraic Graphs with Class}
\author{
  Christoph Madlener\\
  \texttt{\href{mailto:madlener@in.tum.de}{madlener@in.tum.de}}
}

\begin{document}

\maketitle
\begin{abstract}
  ~\cite{mokhov2017algebraic} 
\end{abstract}

\section{Introduction}
\begin{itemize}
\item graph representations (in maths, in Haskell/other programming languages)
  \begin{itemize}
  \item ``standard'' tuple $G=(V,E), E \subseteq V \times V$
  \item containers~\cite{king1995dfs}, fgl~\cite{erwig2001inductive} (state of the art Haskell libraries)
  \end{itemize}
\item wellformedness (\textrightarrow{} implicit/explicit vertex set (?),
  partial functions)
\item introduce datatype (?)
\item sound and complete representation for graphs
\item overview
\end{itemize}

\section{Type class/locale}
\subsection{The Core}
\begin{itemize}
\item graph construction primitives
\item abstract from datatype \textrightarrow{} type class
\item fundamental functions (vertex, edge, etc.)
\end{itemize}
\subsection{Algebraic Structure}
\begin{itemize}
\item axioms (for digraphs)
\item subgraph - partial ordering
\item axioms for undirected graphs, other classes
\item instances in Haskell (directed, undirected, etc. via Eq instance)
\end{itemize}
\subsection{Graph Transformation Library}
\begin{itemize}
\item graph families (path, circuit, etc.)
\item transpose
\item functor
\item monad
\item removing edges
\end{itemize}
\subsection{Verification}
\begin{itemize}
\item equational reasoning
\item reasoning in Isabelle/HOL \textrightarrow{} locale
  \begin{itemize}
  \item ``advanced'' graph concepts (walks, reachability, SCCs, etc.)
    \begin{itemize}
    \item more experimentation needed for actual feasibility (additional axioms required
      for e.g.\ $Vertex\ u \neq \epsilon$?)
    \item first impression: similar as with other representations; abstraction
      to more advanced concepts quickly hides representation
    \end{itemize}
  \item ``polymorphic'' lemmas
  \item compare to Kruskal AFP entry~\cite{Kruskal-AFP} (possibility to get algorithms for
    any graph representation which can instantiate locale)
  \item instantiation with pair\_digraph~\cite{GraphTheory-AFP} - soundness and
    completeness proof
  \end{itemize}
\end{itemize}

\section{Deep Embedding}
\begin{itemize}
\item compact representation
\item compare to different concrete/executable representations wrt.\ efficiency
  in time and space 
  \begin{itemize}
  \item sparse graphs (edge list)
  \item adjacency map
  \end{itemize}
\item performance (\href{https://github.com/haskell-perf/graphs}{haskell-perf})
\item reasoning in Isabelle/HOL \textrightarrow{} induction is great
\item algorithms directly on deep embedding? \textrightarrow{} future work/open
  question (Haskell implementation converts to adjacency map for algorithms)
\end{itemize}
\subsection{Quotient Type}
\begin{itemize}
\item deep embedding which satisfies axioms
\item additional lemmas
\item minimization~\cite{mcconnell2005linear}
  \begin{itemize}
  \item interesting for quotient type, disconnect from Noschinski/any other
    representation for equality, same applies in Haskell for Eq instance)
    \item interesting for algorithms (assuming we can exploit compact
      representation)
    \end{itemize}
\end{itemize}

\section{Conclusion}
\begin{itemize}
\item summarize findings
\item restate open questions, future work
\item labeled graphs (outlook)
\item related work: other algebraic approaches (semiring on
  matrices~\cite{dolan2013fun}, relational
  algebra~\cite{berghammer2020relational})
\item usage examples~\cite{mokhov2019language,beaumont2017high}
\end{itemize}
\printbibliography
\end{document}